\section{Final Exam Prep}
\subsection{Problem 1: True, False, or Unknown (20 pts)}
Indicate T, F, or U.

\begin{enumerate}
    \item If a language is in P, then it is also in NP. - True. all languages in P can be decided in polynomial time and thus they can also be verified. We can simply create a verifier that ignores the certificate, runs the decider and output what the decider outputs.
    \item The complement of an undecidable language must be undecidable. false
    \item If $\text{SAT} \le_p L$ and $L$ is decidable in polynomial time, then $P = NP$. True
    \item Every infinite regular language has an infinite complement. - true
    \item If $M$ is a Turing machine that halts on all inputs, then $L(M)$ is decidable. - True
    \item If $A \le_p B$ and $B$ is regular, then $A$ must be regular.
    \item The language $\{\langle M \rangle : M \text{ halts on at least one input}\}$ is decidable. - false
    \item For any languages $A$ and $B$, if $A \cup B$ is decidable, then both $A$ and $B$ are decidable. false
    \item If a language is NP-complete, then its complement is also NP-complete. 
\end{enumerate}

\subsection{Problem 2: Short Answer (15 pts)}

\begin{enumerate}
    \item Construct an NFA for the language over $\Sigma = \{0,1\}$ consisting of all strings that end with $010$.
    \item Give a context-free grammar for the language 
    \[
        L = \{0^n 1^m : n \le m\}.
    \]
    \item Define polynomial-time many-one reducibility and explain why such reductions preserve NP-completeness.
\end{enumerate}

\subsection{Problem 3: Undecidability (10 pts)}

Define
\[
H_{\infty} = \{\langle M \rangle : M \text{ halts on infinitely many inputs}\}.
\]
Show that $H_{\infty}$ is undecidable. You must reduce from a known undecidable language (such as $HALT$ or $A_{TM}$). Do not use Rice's Theorem.

By contradiction decider $D_{ATM} \{$
1. define

\subsection{Problem 4: NP-Completeness (10 pts)}

Define the decision problem \textsc{Even-Path-Sum}:

\begin{quote}
\textbf{Input:} A directed graph $G = (V,E)$, a weight function $w : E \to \mathbb{Z}$, and two vertices $s,t \in V$, and a target integer $k$.  
\textbf{Output:} YES iff:
\begin{enumerate}
    \item every edge on every $s$--$t$ path has even weight, and
    \item there exists at least one $s$--$t$ path whose total weight is exactly $k$.
\end{enumerate}
\end{quote}

\begin{enumerate}
    \item[(a)] Prove that \textsc{Even-Path-Sum} is in NP.
    \item[(b)] Prove that \textsc{Even-Path-Sum} is NP-complete by giving a polynomial-time reduction from a known NP-complete problem.
\end{enumerate}
\textbf{A:}
\\
Certificate: 
\newpage
\subsection{Summary of Post-Midterm 2 Topics}

\textbf{Runtime of Turing Machines.}
The runtime of a Turing machine on input $x$ is the number of steps taken before halting.  
For inputs of length $n$, the worst-case runtime is described by a function $t(n)$.



\textbf{TIME$(t(n))$.}
The class $\mathrm{TIME}(t(n))$ consists of all languages decidable by a deterministic
Turing machine in $O(t(n))$ time.



\textbf{P.}
$P$ is the class of decision problems solvable in polynomial time on deterministic
Turing machines.



\textbf{NP.}
$NP$ is the class of languages whose membership can be verified in polynomial time given
a suitable certificate. A language is in $NP$ if there exists a polynomial-time deterministic verifier.



\textbf{Polynomial-Time Reductions.}
A language $A$ polynomial-time reduces to $B$ (written $A \le_p B$) if there exists a
polynomial-time computable function $f$ such that
\[
x \in A \iff f(x) \in B.
\]
Such reductions preserve computational hardness.



\textbf{Polynomial-Time Mapping Reductions.}
These are the standard many-one reductions used in complexity theory: a single
polynomial-time transformation maps instances of one problem to another while preserving
YES/NO answers.



\textbf{NP-Completeness.}
A language $L$ is NP-complete if:
\begin{enumerate}
    \item $L \in NP$, and
    \item for every language $A \in NP$, $A \le_p L$.
\end{enumerate}
NP-complete problems are the hardest problems in $NP$.



\textbf{SAT.}
The Boolean satisfiability problem asks whether a Boolean formula has a truth assignment
that makes it true. SAT is NP-complete.



\textbf{3SAT.}
A restricted form of SAT in which the formula is in CNF and every clause contains
exactly three literals. 3SAT is NP-complete and serves as a central source of reductions.



\textbf{CLIQUE.}
Given a graph $G$ and integer $k$, determine whether $G$ contains a clique of size $k$.
CLIQUE is NP-complete via a reduction from 3SAT.



\textbf{SUBSET-SUM.}
Given integers $a_1,\dots,a_n$ and a target $t$, the problem asks whether some subset
sums to $t$. SUBSET-SUM is NP-complete.



\textbf{3COLOR.}
Given a graph, determine whether its vertices can be colored using three colors so that
adjacent vertices receive different colors. 3COLOR is NP-complete via reductions from 3SAT.



\textbf{Cook--Levin Theorem.}
This theorem establishes that SAT is NP-complete by showing that every nondeterministic
polynomial-time computation can be encoded as a Boolean formula whose satisfiability
captures exactly the accepting computations of the machine.

\newpage
\subsection{Additional Reference Material}

\textbf{Closure Properties of Complexity Classes.}
\begin{itemize}
    \item $P$ is closed under union, intersection, complement, concatenation, and Kleene star.
    \item $NP$ is closed under union and concatenation; complement closure is unknown.
    \item Regular languages and context-free languages are closed under standard Boolean operations (except CFLs are not closed under intersection or complement).
\end{itemize}



\textbf{Undecidability Results.}
\begin{itemize}
    \item $HALT = \{\langle M,w\rangle : M \text{ halts on } w\}$ is undecidable.
    \item $A_{TM} = \{\langle M,w\rangle : M \text{ accepts } w\}$ is undecidable.
    \item $HALT_{TM}$, $EQ_{TM}$, $REGULAR_{TM}$, and $\{\langle M\rangle : L(M) = \emptyset\}$ are undecidable.
    \item Reductions from $HALT$ or $A_{TM}$ are the standard method to prove undecidability.
\end{itemize}

\textbf{Standard Techniques for Proving Undecidability.}
\begin{itemize}
    \item Construct a Turing machine $N$ that simulates $M$ on a fixed input.
    \item Embed the behavior of $M$ into $N$ so that questions about $N$ encode questions about $M$.
    \item Flip acceptance and rejection or force halting to control the target language.
\end{itemize}



\textbf{Useful Definitions.}
\begin{itemize}
    \item A \emph{decider} always halts on all inputs.
    \item A language is \emph{recognizable} if some TM accepts exactly its strings.
    \item A language is \emph{co-recognizable} if its complement is recognizable.
    \item A function $f$ is a polynomial-time reduction if $f$ is computable in polynomial time and $x \in A \iff f(x) \in B.$
\end{itemize}



\textbf{Reductions You Should Know.}
\begin{itemize}
    \item $SAT \le_p 3SAT$
    \item $3SAT \le_p \textsc{CLIQUE}$
    \item $3SAT \le_p \textsc{3COLOR}$
    \item $3SAT \le_p \textsc{Vertex Cover}$
    \item $\textsc{SUBSET-SUM} \le_p \textsc{Partition}$
\end{itemize}

\textbf{General Strategy for NP-Completeness Proofs.}
\begin{enumerate}
    \item Show the language is in $NP$ (certificate, verifier, runtime).
    \item Reduce from a known NP-complete problem, typically 3SAT.
    \item Construct a polynomial-time function mapping instances. Prove correctness (see above)
\end{enumerate}

To show L is not regular, show: $\forall p \in \mathbb{N}, \exists \text{ strings } w \in L$ of length $|w| \geq P, \forall \text{ strings } x,y,z \: : \: w = xyz, |y| > 0, |xy| \leq p, \exists i \in \mathbb{N}, x y^i z \notin L$