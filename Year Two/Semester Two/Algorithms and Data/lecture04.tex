\section{January 20, Divide and Conquer cont.}
Quiz 1 1/21 - Basic iterative and recursive algorithms (study first 2 lectures). Practice quiz on canvas.
\\
Homework 1 due on friday.
\subsection{Asymptotic Analysis cont.}
\begin{question}
  Rank the following functions in order of growth:
  \begin{enumerate}
    \item $n \log_2 n$
    \item $n^2$
    \item $100n$
    \item $3^{\log_2 n}$
  \end{enumerate}
  Starting with the first 2. 
  $\lim_{n \to \infty}\frac{\log_2n}{n} = \lim_{n \to \infty}\frac{\ln n}{\ln 2 \cdot n}$.
  \\
  Apply L'Hopital
  \\
  $\lim_{n \to \infty}\frac{1/n}{\ln 2} = \lim_{n \to \infty}\frac{1}{n \cdot \ln 2} = 0. \\
  \therefore \log_2 n = O(n)$
  \\
  $3^{\log_2 n} = (2^{\log_2 3})^{\log_2 n} = (2^{\log_2 n})^{\log_2 3} = n ^{\log_2 3} = n^{1.9}$
  \\
  \textbf{Answer:}
  \begin{enumerate}
    \item $100n$
    \item $n \log_2 n$
    \item $3^{\log_2 n}$
    \item $n^2$
  \end{enumerate}
\end{question}
h
\\
Big Oh Rules:
\begin{itemize}
  \item Constant factors can be ignored. $\forall C > 0, Cn = \mathcal{O}(n)$
  \item Lower order terms can be dropped. $n^2 + n^{3/2} + n = \mathcal{O}(n^2)$
  \item Smaller exponents are Big-Oh of larger exponents. $\forall a < b, n^a = \mathcal{O}(n^b)$
  \item Any logarithm is Big-Oh of any polynomial. $\forall a, b > 0, (log_2 n)^a = \mathcal{O}(n^b)$
  \item Any polynomial is Big-Oh of any exponential. $\forall a > 0, b > 1, n^a = \mathcal{O}(b^n)$
  \item Bases of logarithms can be ignored. $\forall a, b > 1, \log_a (n) = \mathcal{O}(\log_b (n))$
\end{itemize}
\begin{itemize}
  \item Big-Omega
\end{itemize}