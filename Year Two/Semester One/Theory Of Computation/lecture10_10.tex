\section{October 10}
\subsection{Language of a TM}
\begin{itemize}
  \item A TM M on input w can either accept, reject, or loop
  \item For a TM M, we define $L(M) = \{w | M \text{ accepts } w \}$
  \item If TM M and a language L satisfies "for any $x \in L$, M accepts x", we cannot say "M recognizes L". We must prove that $L(M) = \{w | M \text{ accepts } w \}$ 
  \item \quad Do this using $\subseteq$ (M accepts any string in L) and $\supseteq$ (If a string is accepted by M, the string is in L).
  \item We say the M \textbf{decides} L if 
  \begin{itemize}
    \item M accepts $w \in L$ and M rejects $w \notin L$.
    \item equivalently: M recognizes L and M always halts.
  \end{itemize}
  \item A language L is \textbf{recognizable} (resp. \textbf{decidable}) if there is some TM that recognizes (resp. decides) L.
  \item The set of all languages decided by turing machines is a subset of all languages recognized by turing machines.
\end{itemize}

\subsection{Specifiying a Turing Machine}
Instead of drawing a state diagram, we give a "tape-head" level description that abstracts out the states/transitions via pseudocode.
\begin{itemize}
  \item Imagine tape-head has small local memort which is "fixed" and cannott grow with input size (states of TM).
  \item Describe how the tape-head should \textbf{walk across the tape} and \textbf{what it should write}.
\end{itemize}
\begin{example}
  $L = \{a^{2n} | n \geq 0\}$ all powers of 2.
  \\
  Walk tape-head from left to right and cross out any other a.
  \begin{itemize}
    \item If tape contained a single 0, accept.
    \item Else if number of 0s was odd, reject.
    \item Else return to the left-hand end of tape, repeat. 
  \end{itemize}
\end{example}
\begin{example}
  $L = \{ w \# w: w \in (0,1)^{*} \}$
  \begin{itemize}
    \item Check input is of form $\{0, 1\}^* \# \{0, 1\}^*$ and reject otherwise.
    \item 
  \end{itemize}
\end{example}

\subsection{Beyond Boolean Functions}
We can also consider TM's that output more than just "accept/reject"
\\
Idea: define the output of a TM as the contents of its tape when it enters a halt state.
\\
A TM M computes a function $f: \: \Sigma^* \to \Sigma^*$ if on every input $w \in \Sigma^*$ the TM \textbf{halts} and its tape contains f(w).

\begin{definition}
  We say that f is \textbf{computable} if some TM M computes it.
\end{definition}
\begin{example}
  $f: \: \{0, 1\}^* \to \{0, 1\}^*$: f(binary rep of n) = binary rep of n + 1. f is the binary successor function.
  \\ 
  Show that this is computable.
  \\
  Given any input string such as '01101101111':
  \\
  Move to the end of the string and continue left until the beginning. Flip all 1's until the first 0, flip the first 0.
  \\
  To prevent crashing off the beginning of the tape if all ones: convert first character to a \# if it is a 1 and add a zero on the end if it is reached.
\end{example}