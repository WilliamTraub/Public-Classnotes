\section{December 2}
\subsection{SUBSET-SUM}
\begin{theorem}
  $SUBSET \mydash SUM \in P \implies 3SAT \in P$
\end{theorem}
Proof outline:
\\ We give TM $R$ that on input $\phi$:
\begin{itemize}
  \item computes numbers $a_1, a_2, \dots, a_n, t$ such that 
  $$\phi \in 3SAT \iff (a_1, a_2, \dots, a_n, t) \in SUBSET\mydash SUM$$
\end{itemize}
We can use binary encoding of boolean values.
\begin{example}
  $$\phi = (x \lor y \lor z) \land (\neg x \lor \neg y \lor z) \land (x \lor y \lor \neg z)$$
  3 variables + 3 clauses (v = 3, c = 3) $\Rightarrow$ 6 digits for each number. For each clause in $C$, include two occurances of $a_c = 1$ in C's digit and 0 in others. Set t = 1 in the first $v$ digits and 3 in the rest $k$ digits
  \begin{center}
    \begin{tabular}{c c c c c c c}
      & x & y & z & 1 & 2 & 3 \\
      \hline
      $a_x^T =$ & 1 & 0 & 0 & 1 & 0 & 1 \\
      $a_x^F =$ & 1 & 0 & 0 & 0 & 1 & 0 \\
      $a_y^T =$ & 0 & 1 & 0 & 1 & 0 & 1 \\
      $a_y^F =$ & 0 & 1 & 0 & 0 & 1 & 0 \\
      $a_z^T =$ & 0 & 0 & 1 & 1 & 1 & 0 \\
      $a_z^F =$ & 0 & 0 & 1 & 0 & 0 & 1 \\
      $2 \cdot a_{c1} =$ & 0 & 0 & 0 & 1 & 0 & 0 \\
      $2 \cdot a_{c2} =$ & 0 & 0 & 0 & 0 & 1 & 0 \\
      $2 \cdot a_{c3} =$ & 0 & 0 & 0 & 0 & 0 & 1 \\
      $t =$ & 1 & 1 & 1 & 3 & 3 & 3 \\
    \end{tabular}
  \end{center}
\end{example}
\begin{proof}
  $(\Rightarrow)$ suppose $\phi$ has satisfying assingment, pick $a_x^T$ if x is true, else $a_x^F$. The sum of these numbers yield 1 in the first $v$ digits because $a_x^T, a_x^F$ have 1 in x's digit and 0 in others, and 1, 2, 3 in the last k digits.
\end{proof}
\newpage
\subsection{Optimization Problems}
\begin{example}
  The knapsack problem: REDUCE SUB-SUM to knapsack
  List = (value$_1$, weight$_1$), ... ,(value$_n$, weight$_n$)
\end{example}
\begin{theorem}
  Cook-Levin Theorem: $3SAT \in P \implies P = NP$.

  \begin{lemma}
    Every function $f : \{0, 1 \}^m \to \{0, 1 \}$ can be represented by a CNF formula $\phi$ of size $\leq m \cdot 2^m$.
    \\
    \textbf{Proof:} For each value that makes f evaluate to 0, add a clause to ensure variables can't take that value.
    \begin{example}
      $f(x_1 = 1, x_2 = 0, x_3 = 1, \dots , x_m = 0) = 0$
      \\
      Add the clause $(\neg x_1 \lor x_2 \lor \neg x_3 \lor \dots \lor x_m)$
    \end{example}
  \end{lemma}
\end{theorem}