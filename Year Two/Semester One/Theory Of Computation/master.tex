\documentclass[12pt]{article}

\usepackage[normalem]{ulem}
\newcommand{\mydash}{\hbox{\sout{ }}} 
\renewcommand{\arraystretch}{1.25}

\usepackage{amsmath, amssymb, amsthm}
\usepackage{tikz}
\usepackage{hyperref}
\usepackage[a4paper, margin=1in]{geometry}
\usepackage{tikz}
\usepackage{amsfonts}
\usepackage{latexsym}
\usepackage{array}
\usepackage{listings}
\usepackage{color}

\definecolor{dkgreen}{rgb}{0,0.6,0}
\definecolor{gray}{rgb}{0.5,0.5,0.5}
\definecolor{mauve}{rgb}{0.58,0,0.82}

\lstset{frame=tb,
  language=Java,
  aboveskip=3mm,
  belowskip=3mm,
  showstringspaces=false,
  columns=flexible,
  basicstyle={\small\ttfamily},
  numbers=none,
  numberstyle=\tiny\color{gray},
  keywordstyle=\color{blue},
  commentstyle=\color{dkgreen},
  stringstyle=\color{mauve},
  breaklines=true,
  breakatwhitespace=true,
  tabsize=3
}


\title{Theory of Computation Notes}
\author{William Traub}
\date{}

\newtheorem{corollary}{Corollary}
\newtheorem{definition}{Definition}
\newtheorem{theorem}{Theorem}
\newtheorem{lemma}[theorem]{Lemma}
\newtheorem{example}{Example}
\newtheorem{observation}{Observation}
\newtheorem{question}{Question}


\begin{document}
\maketitle
\tableofcontents

%\section{September 9}

\begin{definition}
A function $f : D \to R$ has domain $D$ and range $R$. Each input $x \in D$ is mapped to exactly one output $f(x) \in R$.
\end{definition}

\begin{example}
The function $\text{add} : \mathbb{Z} \times \mathbb{Z} \to \mathbb{Z}$ is defined by
\[
\text{add}(x,y) = x + y.
\]
\end{example}

\subsection*{Goal of Computation}
We focus on computing functions $f : \Sigma^* \to \{\text{accept}, \text{reject}\}$.
\begin{itemize}
    \item \textbf{Domain:} strings over alphabet $\Sigma$.
    \item \textbf{Range:} Boolean $\{0,1\}$ or $\{\text{accept}, \text{reject}\}$.
\end{itemize}

Why strings? Any input can be encoded as a string.  
Why booleans? Simplicity, while still capturing many interesting functions.

\subsection*{Functions as Languages}
A language $L$ over $\Sigma$ is a subset of $\Sigma^*$.  
Example: $L = \{ w \in \{0,1\}^* : w \text{ ends with } 1\} = \{1, 01, 11, 001, 101, \dots\}$.

Equivalence between functions and languages:
\[
f \leftrightarrow L \quad \text{where} \quad
L = \{w : f(w) = \text{accept}\}.
\]

\subsection*{Observation}
Languages may be finite or infinite, but a ``program'' is always a finite description.

\section*{Finite Automata}
A \textbf{deterministic finite automaton (DFA)} consists of:
\begin{itemize}
    \item States (nodes).
    \item Transitions labeled by alphabet symbols.
    \item Unique start state $q_0$.
    \item Accept states (double circles).
\end{itemize}

\begin{definition}
A DFA is a 5-tuple $M = (Q, \Sigma, \delta, q_0, F)$ where:
\begin{itemize}
    \item $Q$ = finite set of states
    \item $\Sigma$ = alphabet
    \item $\delta : Q \times \Sigma \to Q$ = transition function
    \item $q_0 \in Q$ = start state
    \item $F \subseteq Q$ = accepting states
\end{itemize}
\end{definition}

\begin{definition}
The extended transition function $\delta^* : Q \times \Sigma^* \to Q$ is defined by:
\[
\delta^*(q,\epsilon) = q, \quad
\delta^*(q, w a) = \delta(\delta^*(q, w), a).
\]
\end{definition}

%\newpage
%\section{September 12}

\begin{theorem}
If $A$ is regular, then so is its complement $A^c$.
\end{theorem}

\begin{definition}
A nondeterministic finite automaton (NFA)
\\
 is a tuple $M = (Q, \Sigma, \gamma, q_{\text{start}}, F)$ where transitions may be nondeterministic or labeled with $\epsilon$.
\end{definition}

An NFA accepts $w$ if there exists some computation path leading to an accept state.

%\newpage
%\section{1/12}
\begin{lstlisting}
  let x = 19 in let x = x < 10 in x ;;
  (* goes to *)
  
  
  let x = 3 in x + x ;; (* -> 6 *)
  let x = 4 in let y = 5 in x * y ;; (* 20 *)
  let x = 19 in (let x = x < 10 in x) ;;
\end{lstlisting}

\subsection{Functions}
Functions are boring and everywhere.
\begin{lstlisting}
  (* anonymous function *)
  let f = fun (x : int) -> x + 1 ;;
  f : int -> int -> int
\end{lstlisting}
OCaml is right-associative so 
\begin{lstlisting}
  let f = fun (x : int) -> fun (y : int) -> x * y ;;
  let f (x : int) (y : int) = x * y ;;
\end{lstlisting}
These functions are isomorphic and will result in about the same thing.
\begin{lstlisting}
  let f (g : int -> int) (y : int) = g y;;
  let adder = f (fun (y : int) )
\end{lstlisting}
Lexical Scope
\begin{lstlisting}
  let y   = 1 in
  let f x = x + y in
  let y   = 2 in 
  f 3 ;;
\end{lstlisting}
There is closure used where the function substitutes y into its body and just becomes f(x) = x + 1.

\subsection{Specification}
We have high-order functions that can take in other functions.
\begin{lstlisting}
  let max3 (i : int) (j : int) (k : int) = 
    if i >= j
    then if i >= k
      then i
      else k
    else if l >= k
      then j
      else k
\end{lstlisting}
We want to specify the function in formal logic.
\begin{multline*}
  \forall i, j, k \in \mathbb{Z} \\
    max3(i, j, k) \geq i \land \\
     max3(i,j,k) \geq j \land \\
     max3(i,j,k) \geq k \land \\
     max3(i,j,k) \in \{i, j, k\}
\end{multline*}

%\newpage
%\section{January 20, Divide and Conquer cont.}
Quiz 1 1/21 - Basic iterative and recursive algorithms (study first 2 lectures). Practice quiz on canvas.
\\
Homework 1 due on friday.
\subsection{Asymptotic Analysis cont.}
\begin{question}
  Rank the following functions in order of growth:
  \begin{enumerate}
    \item $n \log_2 n$
    \item $n^2$
    \item $100n$
    \item $3^{\log_2 n}$
  \end{enumerate}
  Starting with the first 2. 
  $\lim_{n \to \infty}\frac{\log_2n}{n} = \lim_{n \to \infty}\frac{\ln n}{\ln 2 \cdot n}$.
  \\
  Apply L'Hopital
  \\
  $\lim_{n \to \infty}\frac{1/n}{\ln 2} = \lim_{n \to \infty}\frac{1}{n \cdot \ln 2} = 0. \\
  \therefore \log_2 n = O(n)$
  \\
  $3^{\log_2 n} = (2^{\log_2 3})^{\log_2 n} = (2^{\log_2 n})^{\log_2 3} = n ^{\log_2 3} = n^{1.9}$
  \\
  \textbf{Answer:}
  \begin{enumerate}
    \item $100n$
    \item $n \log_2 n$
    \item $3^{\log_2 n}$
    \item $n^2$
  \end{enumerate}
\end{question}
h
\\
Big Oh Rules:
\begin{itemize}
  \item Constant factors can be ignored. $\forall C > 0, Cn = \mathcal{O}(n)$
  \item Lower order terms can be dropped. $n^2 + n^{3/2} + n = \mathcal{O}(n^2)$
  \item Smaller exponents are Big-Oh of larger exponents. $\forall a < b, n^a = \mathcal{O}(n^b)$
  \item Any logarithm is Big-Oh of any polynomial. $\forall a, b > 0, (log_2 n)^a = \mathcal{O}(n^b)$
  \item Any polynomial is Big-Oh of any exponential. $\forall a > 0, b > 1, n^a = \mathcal{O}(b^n)$
  \item Bases of logarithms can be ignored. $\forall a, b > 1, \log_a (n) = \mathcal{O}(\log_b (n))$
\end{itemize}
\begin{itemize}
  \item Big-Omega
\end{itemize}
\newpage
\section{October 3}
\subsection{Turing Machines}
"If you want to learn anything about automata you can just ask chatGPT"
\\
A Turing machine is a General Model of Computation
\begin{itemize}
    \item \textbf{Algorithms have been around since dawn of time.}
          \begin{itemize} 
            \item Long addition, multiplication, division.
            \item Compass and straightedge constructions 
            \item Euclid's greatest common divisor algorithm  
            \item Quadratic formula: finding roots of polynomials
          \end{itemize}
    \item Traditionally, algorithms were understood as a human construct. No precise mathematical definition.  
\end{itemize}
Already saw a limited notion of algorithms (DFA). Using the pumping lemma, we proved that there are some problems that are not computable in this model.
\subsubsection{David Hilbert's Descision Problem}
In 1928, David Hilbert asked for an "algorithm" that takes as input a mathemattical statement and decides whether the statement is \underline{true} or \underline{false}.
\\
During the years 1931-1936, a series of works showed there is no algorithm for the decision problem.
\\
Each of these works included a different definition of a \underline{"general algorithm"}.
\begin{itemize}
  \item Kurt Godel relied on recursive functions.
  \item Alonzo Church developed $\lambda$-calculus.
  \item Alan Turing developed the Turing Machine.
\end{itemize}
All of these definitions turn out to be equivalent.
\\
Turing Machines are perhaps the most intuitive. They provided inspiration for a general computer, the \underline{Von Neumann Architecture}
\newpage
\subsubsection*{Turing Machines cont.}
Our Plan
\begin{itemize}
  \item Define Turing Machines (TM). See how they work.
  \item Convince ourselves that TMs are powerful enough to implement any "reasonable algorithm".
    
\end{itemize}
A TM is like a DFA with infinite memory tape. Information can be saved and accessed using the tape instead of the DFA's state space.
\begin{itemize}
  \item Initially, tape contains the input, followed by "blanks". The tape head is at the left-most position.
  \item In each step, the machine can overwrite the symbol under the tape-head and move the tape left or right.
    \begin{itemize}
      \item The tape head cannot move left of the start.
      \item TMs can use additional symbols to write to tape.
    \end{itemize}
  \item At any point in time, the machine can halt the computation and \underline{accept} or \underline{reject}. (If there is no decision edge at the state the head is on, also reject)
  \item This is implemented via states and transitions like a DFA
\end{itemize}
\begin{definition} \label{TuringMachine}
  A \underline{Turing Machine} consists of a tuple: 
  \\$M = (Q, \Sigma, \Gamma, \delta, q_{start}, q_{accept}, q_{reject}$) Where
  \begin{itemize}
    \item Q is a finite set called the \underline{states}. 
    \item $\Sigma$ is an \underline{input alphabet}.
    \item $\Gamma$ is the \underline{tape alphabet} such that $\Sigma \subseteq \Gamma$ and $\Gamma$ contains a special blank symbol '-' that is not in $\Sigma$.
    \item $q_{start} \in Q$ is the \underline{start state}.
    \item $q_{accept} \in Q$ is the \underline{accept state}, $q_{reject}$ is the reject state.
    \item $\delta : Q' \times \Gamma \rightarrow Q \times \Gamma \times \{L, R\}$ is the \underline{transition function}.
    \\ \indent \indent Where $Q' = Q \setminus \{q_{accept}, q_{reject}\}$
  \end{itemize}
\end{definition}


\newpage
\section{October 7}
\subsection{Building a Turing Machine}
\begin{definition}
  \textbf{Configuration} encodes all information about a particular step in the computation of a turing machine.
  \\
  All information:
  \begin{itemize}
    \item Current state
    \item Content on the tape
    \item Tape-head position
  \end{itemize}
  Let $M = (Q, \Sigma, \Gamma, \delta, q_{start}, q_{accept}, q_{reject}$ be a Turing Machine.
  \begin{itemize}
    \item A \textbf{configuration} of M is a tuple $C=(u,q,v)$ such that $u,v \in \Gamma^{*}$ and $q \in Q$. Can write $C=uqv$ without commas.
    \item A configuration $C$ \textbf{yields} $C'$ if M goes from C to C' in 1 step.
    \begin{itemize}
      \item $C = (u, q, bw)$ yields $C' = (ub', q', w)$ if $\delta(q,b) = (q', b', R)$
      \item $C = (ua, q, bw)$ yields $C' =  (u,q',ab'w)$ if $\delta(q,b) = (q', b', L))$
      \item $C = (q,bw)$ yields $C' = (q',b'w)$ if $(q,b) = (q', b', L)$ (don't fall off)
    \end{itemize}
    \item A start configuration of M on input W is $q_{start}W$
    \item An accepting (resp.  rejecting) configuration is one where the state is $q_{accept}$ (resp.  $q_{reject}$). 
  \end{itemize} 
  M accepts (resp. rejects) w if there is a sequence of configurations $C_1, C_2, ... , C_n$ such that:
  \begin{itemize}
    \item $C_1$ is the start configuration of M on input w.
    \item $C_i$ yields $C_{i+1}$ for $i = 1,...,n-1$.
    \item $C_n$ is an accepting (resp. rejecting) configuration.
  \end{itemize}
\end{definition}
This is a way to save your current state for later.
\\
If $C = (ua, q, bv), \: \delta(a, b) = (q', c, L)$, then $C' = (u, q', acv)$
\\
If $\delta(a, b) = (q', C, R)$ then $C' = (uac, q', v)$

\newpage
\section{October 10}
\subsection{Language of a TM}
\begin{itemize}
  \item A TM M on input w can either accept, reject, or loop
  \item For a TM M, we define $L(M) = \{w | M \text{ accepts } w \}$
  \item If TM M and a language L satisfies "for any $x \in L$, M accepts x", we cannot say "M recognizes L". We must prove that $L(M) = \{w | M \text{ accepts } w \}$ 
  \item \quad Do this using $\subseteq$ (M accepts any string in L) and $\supseteq$ (If a string is accepted by M, the string is in L).
  \item We say the M \textbf{decides} L if 
  \begin{itemize}
    \item M accepts $w \in L$ and M rejects $w \notin L$.
    \item equivalently: M recognizes L and M always halts.
  \end{itemize}
  \item A language L is \textbf{recognizable} (resp. \textbf{decidable}) if there is some TM that recognizes (resp. decides) L.
  \item The set of all languages decided by turing machines is a subset of all languages recognized by turing machines.
\end{itemize}

\subsection{Specifiying a Turing Machine}
Instead of drawing a state diagram, we give a "tape-head" level description that abstracts out the states/transitions via pseudocode.
\begin{itemize}
  \item Imagine tape-head has small local memort which is "fixed" and cannott grow with input size (states of TM).
  \item Describe how the tape-head should \textbf{walk across the tape} and \textbf{what it should write}.
\end{itemize}
\begin{example}
  $L = \{a^{2n} | n \geq 0\}$ all powers of 2.
  \\
  Walk tape-head from left to right and cross out any other a.
  \begin{itemize}
    \item If tape contained a single 0, accept.
    \item Else if number of 0s was odd, reject.
    \item Else return to the left-hand end of tape, repeat. 
  \end{itemize}
\end{example}
\begin{example}
  $L = \{ w \# w: w \in (0,1)^{*} \}$
  \begin{itemize}
    \item Check input is of form $\{0, 1\}^* \# \{0, 1\}^*$ and reject otherwise.
    \item 
  \end{itemize}
\end{example}

\subsection{Beyond Boolean Functions}
We can also consider TM's that output more than just "accept/reject"
\\
Idea: define the output of a TM as the contents of its tape when it enters a halt state.
\\
A TM M computes a function $f: \: \Sigma^* \to \Sigma^*$ if on every input $w \in \Sigma^*$ the TM \textbf{halts} and its tape contains f(w).

\begin{definition}
  We say that f is \textbf{computable} if some TM M computes it.
\end{definition}
\begin{example}
  $f: \: \{0, 1\}^* \to \{0, 1\}^*$: f(binary rep of n) = binary rep of n + 1. f is the binary successor function.
  \\ 
  Show that this is computable.
  \\
  Given any input string such as '01101101111':
  \\
  Move to the end of the string and continue left until the beginning. Flip all 1's until the first 0, flip the first 0.
  \\
  To prevent crashing off the beginning of the tape if all ones: convert first character to a \# if it is a 1 and add a zero on the end if it is reached.
\end{example}
\newpage
\section{October 14 - Turing Machine Variants}
\subsection*{Multi-Tape TM}
A TM with one input tape and multiple work tapes. Transition function is defined by $\delta = Q' \times T \to Q \times T \times \{L, R\}$.   $\delta(g, w) = (g', w, L/R)$.
\begin{itemize}
  \item You can concatenate multiple tapes to one tape and separate their contents by \#. 
  \item Remember tape-head positions by storing an \underline{underlined} version of tape symbols. 
  \item Each step of multi-tape TM is simulated by scanning entire tape of single-tape TM.
  \item If you run out of space on the tapes, shift all elements to the right. (halting problem)
\end{itemize}
Tape-Head Level Description:
\\
f(a, b) = a + b. Input (binary interpretations of two integers separated by a \#)
\begin{itemize}
  \item Reverse each input and copy each one to a different tape and clear main tape. 
  \item Return all tape heads to the left. Store 1-bit carry as 0
  \item Add two bits under each tape head (using 0 if one head is empty) and carry bit:
    \begin{itemize}
      \item Write result mod 2 to the main tape.
      \item Move all tape heads 1 right.
      \item Repeat until both heads are empty.
    \end{itemize}
  \item Reverse main tape.
\end{itemize}

\subsection*{Random Access TM}
Can read/write to arbitrary locations in memory without scanning a tape. Memory modeled as infinite array R.
\begin{itemize}
  \item In addition to the standard tape that contains the input the TM has location and value tapes.
  \item There is a special write transition which sets R[location] = value using the content of the tapes.
  \item There is a read transition which sets the contents of the values tape to R[location]
\end{itemize}
Compliling to normal TM:
\\
We use a multi-tape machine (which can be converted to a single-tape)
\begin{itemize}
  \item Store contents of array R on a tape as tuples (location$_n$, value$_n$).
  \item To simulate a read, scan R until find a location that matches content of location tape. Write the value on the value tape. Put a blank if no such value is found.
  \item To simulate a write, scan R until you find location that matches content of location tape. Update value. If none found, append (location, value) to end of R.
\end{itemize}
\subsection*{Turing Completeness}
\begin{theorem}
  Church-Turing Thesis: Any algorithm (in an informal sense) can be computed by a TM.
\end{theorem}
Proof Outline:
\\
Design a compiler that converts Java program into a TM.
\begin{itemize}
  \item All programming languages are already compiled to "assembly code" for modern CPUs
  \item Assembly code instructions can be implemented on a Random-Access TM.
\end{itemize}


\newpage
\section{October 21}
For each object O, let $<O>$ be a string that encodes O. 
\begin{itemize}
    \item If i is an integer $<i>$ can be its representation
    \item If O is a string, then $<O>$ is just O itself.
    \item If G is a graph $<G>$ can be defined in many ways; such as vertex lists, etc
    \item If M is a TM, we can define $<M>$ by writing down the formal definition. $M = (Q, \Sigma, \Gamma, \delta, q_{start}, q_{accept}, q_{reject})$
\end{itemize}
Now we will consider more complex languages, i.e. primes, graphs, etc.

\subsection{Universal Turing Machine}
\begin{itemize}
  \item There is a Turing Machine $M_{\text{UNIV}}$ that can run any other TM.
  \item $M_{\text{UNIV}}(<M>, w)$. Takes as input a description of any TM M and any string W. Runs M on w.
  \begin{itemize}
    \item If M accepts w, then $M_{\text{UNIV}}(<M>, w)$ accepts.
    \item If M rejects w, then $M_{\text{UNIV}}(<M>, w)$ rejects.
    \item If M loops on w, then $M_{\text{UNIV}}(<M>, w)$ also loops.
  \end{itemize}
  \item TMs are algorithms and a universal TM is a general purpose computer.
\end{itemize}


\newpage
\section{October 28}
Goal: Show some languages are not decidable.
\subsection{Comparing Infinities}
There are many infinite sets:
\begin{itemize}
  \item $\mathbb{N}$ - Natural numbers
  \item Even numbers
  \item $\mathbb{Q}$ - Rationals
\end{itemize}
Some infinite sets are bigger than others. We can show that one set A is larger than another B if a one-to-one map exists between B and A.
\\
If there is a one-to-one function in both directions, B is the same size as A.
\\
Natural numbers are the "smallest" infinite set, if A is infinite: $|\mathbb{N}| \leq |A| $. An infinite set A is countable if $|A| = |\mathbb{N}|$. We can also show $|A| \leq |\mathbb{N}|$ as there is no set $|S| < |\mathbb{N}|$.

\subsection{Uncountablility}
To show A is uncountable, it is enough to show some set $|B| \leq |A|$ where B is uncountable. 
\\
\textbf{Proof:} Because of the transitivity of the $\leq$ operator, 
\begin{itemize}
  \item The real numbers $\mathbb{R}$ are uncountable: $|\mathbb{S}| \leq |\mathbb{R}|$. The one-to-one function $f: \mathbb{S} \to \mathbb{R}$ defined by: $f(s) = .a_1 a_2 .a_3...$ (in decimal) where $s = a_1, a_2, a_3 ...$
  \item The set $\mathbb{P}$ PowerSet($\mathbb{N}$) is uncountable: $|\mathbb{S}| \leq |\mathbb{P}|$. One-to-one function $f: \mathbb{S} \to \mathbb{P}$ is defined by $f(s) = \{i : a_i = 1 \}$ where $s = a_1, a_2, a_3 ...$
\end{itemize}

\subsection{Undecidability}
We know:
\begin{itemize}
  \item The set $\mathbb{L}$ of all languages is uncountable.
  \item The set $\mathbb{M}$ of all TM's are countably infinite.
\end{itemize}
Because of the previous proof, we know that $|\mathbb{M}| \leq |\mathbb{L}|$
\\
We will show $A_{TM} = \{<M,w> : \text{M is a TM that accepts w} \}$ is undecidable. 
\\
Given a description of a TM M and a string w:
\begin{enumerate}
  \item Decide if M accepts w. (this is $A_{TM}$)
  \item Decide if M halts on the input w.
  \item Decide if M halts on empty input $\varepsilon$
  \item Decide if $L(M) = \emptyset$
\end{enumerate}

\subsection{The TM Self-Acceptance Problem}
Take the turing machine $SA_{TM} = \{<M> : \text{M is a TM that accepts }<M>\}$
\\
The complement of this: $SU_{TM} = \{<M> : \text{M is a TM that does not accept} <M> \}$
\begin{enumerate}
  \item A TM is "self-accepting" if it accepts the string $<M>$ denoting its own description
  \item To decide the language $SA_{TM}$, you need to design an algorithm that gets $<M>$ and decides M is self-accepting.
  \item $SA_{TM}$, $SU_{TM}$ are complements of each other (assume every string denotes some TM). One is decidable $\iff$ the other is.
\end{enumerate}
\begin{theorem}
  \textbf{Claim:} $SU_{TM}$ is an undecidable language.
  \\ \textbf{Proof:} By contradiction, Assume we have a decider D (a TM that always halts for $SU_{TM}$). D accepts $<M> \iff$ M does not accept $<M>$.
  \\
  This can be rewritten as D accepts $<D> \iff$ D does not accept $<D>$, hence we arrive at a contradiction.
\end{theorem}

\subsection{Undecidability as Diagonalization}
Because the set of all TMs are countable, we can create the matrix $[M_i \times <M_i>]$ where the diagonals of the row show when the TM accepts (resp. rejects) itself.
\\
We can use this to directly prove undecidability of self-acceptance.
\\
Suppose by contradiction there exists a decider D for $SA_{TM} = \{<M> \vert \text{ M accepts} <M> \}$. We can construct $M^*(<M>)$: Outputs $\neg D(<M>)$
\\
Consider $M^*(<M^*>)$ (the diagonal position in our matrix). This would result in $M^*(<M^*>) \text{accepts} \iff D(<M^*>) \text{rejects} \iff \neg (M^*(<M^*>) \text{accepts})$. This results in a contradiction and therefore a counterexample.

\subsection{Reductions in Undecidability}
The TM Acceptance problem: $A_{TM} = \{<M,w> : \text{M is a TM that accepts w} \}$.
\\
We previously showed that $SA_{TM} = \{<M> \vert \text{ M accepts} <M> \}$ is undecidable. If we had a decider $D_A$ for $A_{TM}$, we could construct a decider $D_S$ for $SA_{TM}$. 
\\
$D_S(<M>) \{ \text{Output }D_A(<M , M>) \}$
\\
Use reduction to solve problems. Reduce problem A to B - show how to solve A given a way to solve B. 
\\
If A and B are languages, than we reduce A to B by constructing decider $D_B$ as a subroutine.
\\
By reducing A to B, we show:
\begin{itemize}
  \item If B is decidable then A is decidable. (algorithms)
  \item If A is undecidable then B is undecidable. (this course)
\end{itemize}

\subsubsection{Halting Problem}
Consider the TM $H = \{ <M, w> : \text{M is a TM and M halts on w} \}$
\begin{theorem}
  \textbf{Claim:} H is undecidable.
  \\ \textbf{Proof:} Reduce $A_{TM}$ to H.
  \\
  Construct decider $D_{ATM}$ using decider $D_H$ as a subroutine.
  \\
  $D_{ATM}(<M, w>) \{$
  \\ \indent Run $D_H(<M,w>)$ and if it rejects, output reject;
  \\ \indent Else run M(w) and output whatever it outputs;
  \\ $\}$
  \\
  Because we know that $D_H$ is a decider $D_{ATM}$ always halts.
  \\
  $\forall <M,w>, <M,w> \in A_{TM} \iff D_{ATM}(<M,w>) \text{accepts} \iff \text{M accepts w} \iff $
\end{theorem}

\newpage
\section{November 7}
\subsection{Proof Systems}
By Russel's Paradox, sets "cannot" be non-wellfounded.
\begin{corollary}
  There is no "proof system" in Number Theory
  \begin{itemize}
    \item Soundness: No false statement $\phi$ has a valid proof $\pi$.
    \item Completeness: Every trye statement $\phi$ has a valid proof $\pi$.
    \item Decidability: Can decide if $\pi$ is a valid proof for $\phi$
  \end{itemize}
\end{corollary}
We generally give up soundness for completeness.
\begin{proof}
  If there was a proof system for number theory, then we could decide number theory.
  \begin{itemize}
    \item Given $\phi$ try all valid strings $\pi$ one-by-one:
    \begin{itemize}
      \item Check if $\pi$ is a valid proof of $\phi$. If so, accept.
      \item Check if $\pi$ is a valid proof of $\neg \phi$. If so, reject.
    \end{itemize}
  \end{itemize}
\end{proof}

\subsection{Gödel's Sentence}
We want to construct a statement in a proof system that is true and unprovable.
\\
Take proposition p(x) = "p(x) is unprovable."
\begin{enumerate}
  \item Proving p(x) is true: $\neg p(x) \implies p(x) \text{ is provable } \implies p(x)$ by the property of soundedness in proof systems. This is clearly a contradiction.
  \item If p(x) is true, it must also be unprovable.
\end{enumerate}
Showing that a proposition can describe itself.
\\
$\Phi^*(n):$
\begin{enumerate}
  \item parse n as $\langle \phi \rangle$
  \item True, if $\nexists$ proof for $\phi(n)$
\end{enumerate}
$\phi^*(n) := \nexists$ proof for $\phi(n)$ where $\langle \phi \rangle = n$.
\\
Consider $\phi^*(\langle \phi^* \rangle)$
\begin{enumerate}
  \item Proving $\phi^*(\langle \phi^* \rangle)$ is true: Suppose not, $\implies \exists$ proof for $\phi^*(\langle \phi^* \rangle)$. By soundedness, we then know that $\phi^*(\langle \phi^* \rangle)$ is true.
  \item Proving $\phi^*(\langle \phi^* \rangle)$ is unprovable: Suppose not, $\implies \phi^*(\langle \phi^* \rangle)$ is true which results in a contradiction.
\end{enumerate}
We can use diagonalization to build a matrix where each row represents the mathematical equation $phi$ and each column represents a natural number n. Each cell represents if there exists a proof for $\phi(n)$.
\newpage
\section{Exam 2 Prep November 13}
\includegraphics[scale=0.5]{build/language_sets.png}
\\
\begin{itemize}
  \item The set $\mathbb{S}$ of all infinite binary sequences is uncountable. We can show a set is uncountable by proving it is larger than $\mathbb{S}$
  \item The TM Acceptance problem: $A_{TM} = \{\langle M,w \rangle : \text{M is a TM that accepts w} \}$.
  \item $A_{TM}$ is recognizable but not decidable. $\overline{A_{TM}}$ is not recognizable.
  \item A language is decidable $\iff$ it is Turing-recognizable and    co-Turing-recognizable.
  \item If A is reducible to B and B is decidable, A also is
  decidable. Equivalently, if A is undecidable and reducible to B, B is undecidable.
  \item By reducing A to B, we show if A is undecidable then B is undecidable.
  \item If $A \leq_m B$ and B is decidable, then A is decidable.
  \item If $A \leq_m B$ and A is decidable, then B is undecidable.
  \item The set of all Turing machines is countable, because every Turing machine has a finite, unique description.
  \item Any Turing-undecidable language is also non-regular, because its contrapositive is true: any regular language is decidable by Turing machines.
  \item Rice's theorem. Let P be any nontrivial property of the language of a Turing machine. Prove that the problem of determining whether a given Turing machine's language has property P is undecidable.
\end{itemize}
\newpage
\section{November 18 - Efficiency (P and NP)}
Are there problems that can be computed, but cannot be computed efficiently? Proving they cannot be computed efficiently using reductions.
\subsection{What is Efficiency?}
If we could compute B efficiently then we could compute A efficiently.
\begin{definition}
  The \textbf{run-time} of a TM M on input w is the number of syeps the TM takes before it halts. 
  This is a function $$f(n) = \text{max} \{\text{run-time of M on w} : |w| = n \}$$
\end{definition}
This is the worst-case runtime of M on input of length n.
\\
Often we don't care abot the exact runtime, but instead the asymptotic runtime.
\begin{example}
  Primality Testing:
  \begin{lstlisting}
    // decide whether q is prime
    boolean IsPrime(q) {
      for (i = 2; i < q; i++) {
        if (q % i == 0) reject and halt
      }
      accept and halt
    }
  \end{lstlisting}
  The runtime of IsPrime is exponential.
\end{example}
\subsection{Asymptotic Notation}
\begin{itemize}
  \item A function $g(n) = O(f(n))$ if there is some constant $c$ such that for all large enough $n: g(n) \leq c f(n)$. Really $O(f(n))$ is a set of functions, $g(n) \in O(f(n))$
  \item Example: $2n^2 + 5n + 7 = O(n^2)$ O only gives an upper bound so this is also $O(n^3)$
\end{itemize}
We define a class of languages:
\\
$TIME(t(n)) = \{L : \exists $ TM M with runtime O(t(n)), M decides L $\}$
\\
Some observations:
\begin{itemize}
  \item $REGULAR \subseteq TIME(n)$
  \item $TIME(2^n) \subseteq DECIDABLE$
\end{itemize}
This classification is reliant on the model used (i.e. a Java program might have O(n) but a TM might be different). Though the difference aren't too big.
\subsection{Polynomial Functions}
\begin{itemize}
  \item A \textbf{Polynomial function} poly(n) = $\cup_c O(n^c) = O(n) \cup O(n^2) \cup O(n^3) \dots$
  
  \item g(n) = poly(n) if and only if there exist constants $c, c'$ such that for all large enough $n: g(n) \leq c n^{c'}$
  \item Composition if f(n), g(n) = poly(n) then $f(g(n)), f(n)g(n), (f(n))^p = poly(n)$
  \item If the runtime of a TM M is some function t(n) = poly(n) then we say that M “runs in polynomial-time”. 
\end{itemize}
The Class $P = \cup \:\: TIME(n^c)$ = the languages that can be decided in time poly(n).
\\
We think of $P$ as the class of langyages that can be decided "efficiently".
\\
The class $P$ is the same if we use any model.
\\
\textbf{Extended Church-Turing thesis}: the class $P$ is the same in any "realistic" model of computation. This is more controversial as the definition of "realisitic" is disputed.
\\
Problems in P
\begin{itemize}
  \item Regular languages
  \item Arithmetic: addition, multiplication, division, exponentiation, etc.
  \item Everything in algorithms class.
\end{itemize}


\newpage
\section{December 2}
\subsection{SUBSET-SUM}
\begin{theorem}
  $SUBSET \mydash SUM \in P \implies 3SAT \in P$
\end{theorem}
Proof outline:
\\ We give TM $R$ that on input $\phi$:
\begin{itemize}
  \item computes numbers $a_1, a_2, \dots, a_n, t$ such that 
  $$\phi \in 3SAT \iff (a_1, a_2, \dots, a_n, t) \in SUBSET\mydash SUM$$
\end{itemize}
We can use binary encoding of boolean values.
\begin{example}
  $$\phi = (x \lor y \lor z) \land (\neg x \lor \neg y \lor z) \land (x \lor y \lor \neg z)$$
  3 variables + 3 clauses (v = 3, c = 3) $\Rightarrow$ 6 digits for each number. For each clause in $C$, include two occurances of $a_c = 1$ in C's digit and 0 in others. Set t = 1 in the first $v$ digits and 3 in the rest $k$ digits
  \begin{center}
    \begin{tabular}{c c c c c c c}
      & x & y & z & 1 & 2 & 3 \\
      \hline
      $a_x^T =$ & 1 & 0 & 0 & 1 & 0 & 1 \\
      $a_x^F =$ & 1 & 0 & 0 & 0 & 1 & 0 \\
      $a_y^T =$ & 0 & 1 & 0 & 1 & 0 & 1 \\
      $a_y^F =$ & 0 & 1 & 0 & 0 & 1 & 0 \\
      $a_z^T =$ & 0 & 0 & 1 & 1 & 1 & 0 \\
      $a_z^F =$ & 0 & 0 & 1 & 0 & 0 & 1 \\
      $2 \cdot a_{c1} =$ & 0 & 0 & 0 & 1 & 0 & 0 \\
      $2 \cdot a_{c2} =$ & 0 & 0 & 0 & 0 & 1 & 0 \\
      $2 \cdot a_{c3} =$ & 0 & 0 & 0 & 0 & 0 & 1 \\
      $t =$ & 1 & 1 & 1 & 3 & 3 & 3 \\
    \end{tabular}
  \end{center}
\end{example}
\begin{proof}
  $(\Rightarrow)$ suppose $\phi$ has satisfying assingment, pick $a_x^T$ if x is true, else $a_x^F$. The sum of these numbers yield 1 in the first $v$ digits because $a_x^T, a_x^F$ have 1 in x's digit and 0 in others, and 1, 2, 3 in the last k digits.
\end{proof}
\newpage
\subsection{Optimization Problems}
\begin{example}
  The knapsack problem: REDUCE SUB-SUM to knapsack
  List = (value$_1$, weight$_1$), ... ,(value$_n$, weight$_n$)
\end{example}
\begin{theorem}
  Cook-Levin Theorem: $3SAT \in P \implies P = NP$.

  \begin{lemma}
    Every function $f : \{0, 1 \}^m \to \{0, 1 \}$ can be represented by a CNF formula $\phi$ of size $\leq m \cdot 2^m$.
    \\
    \textbf{Proof:} For each value that makes f evaluate to 0, add a clause to ensure variables can't take that value.
    \begin{example}
      $f(x_1 = 1, x_2 = 0, x_3 = 1, \dots , x_m = 0) = 0$
      \\
      Add the clause $(\neg x_1 \lor x_2 \lor \neg x_3 \lor \dots \lor x_m)$
    \end{example}
  \end{lemma}
\end{theorem}
\end{document}
