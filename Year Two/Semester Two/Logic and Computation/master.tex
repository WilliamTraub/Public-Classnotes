\documentclass[12pt]{article}

\usepackage[normalem]{ulem}
\newcommand{\mydash}{\hbox{\sout{ }}} 
\renewcommand{\arraystretch}{1.25}

\usepackage{amsmath, amssymb, amsthm}
\usepackage{tikz}
\usepackage{hyperref}
\usepackage[a4paper, margin=1in]{geometry}
\usepackage{tikz}
\usepackage{amsfonts}
\usepackage{latexsym}
\usepackage{array}
\usepackage{listings}
\usepackage{color}
\usepackage{marvosym}
\usepackage{breqn}


\definecolor{dkgreen}{rgb}{0,0.6,0}
\definecolor{gray}{rgb}{0.5,0.5,0.5}
\definecolor{mauve}{rgb}{0.58,0,0.82}

\lstset{
 language=caml,
 columns=[c]fixed,
 basicstyle=\small\ttfamily,
 keywordstyle=\bfseries,
 upquote=true,
 commentstyle=,
 breaklines=true,
 showstringspaces=false,
 stringstyle=\color{blue},
 literate={'"'}{\textquotesingle "\textquotesingle}3
}



\title{Logic and Computation Notes}
\author{William Traub}
\date{}

\newtheorem{corollary}{Corollary}
\newtheorem{definition}{Definition}
\newtheorem{theorem}{Theorem}
\newtheorem{lemma}[theorem]{Lemma}
\newtheorem{example}{Example}
\newtheorem{observation}{Observation}
\newtheorem{question}{Question}


\begin{document}
\maketitle
\tableofcontents

\section{September 9}

\begin{definition}
A function $f : D \to R$ has domain $D$ and range $R$. Each input $x \in D$ is mapped to exactly one output $f(x) \in R$.
\end{definition}

\begin{example}
The function $\text{add} : \mathbb{Z} \times \mathbb{Z} \to \mathbb{Z}$ is defined by
\[
\text{add}(x,y) = x + y.
\]
\end{example}

\subsection*{Goal of Computation}
We focus on computing functions $f : \Sigma^* \to \{\text{accept}, \text{reject}\}$.
\begin{itemize}
    \item \textbf{Domain:} strings over alphabet $\Sigma$.
    \item \textbf{Range:} Boolean $\{0,1\}$ or $\{\text{accept}, \text{reject}\}$.
\end{itemize}

Why strings? Any input can be encoded as a string.  
Why booleans? Simplicity, while still capturing many interesting functions.

\subsection*{Functions as Languages}
A language $L$ over $\Sigma$ is a subset of $\Sigma^*$.  
Example: $L = \{ w \in \{0,1\}^* : w \text{ ends with } 1\} = \{1, 01, 11, 001, 101, \dots\}$.

Equivalence between functions and languages:
\[
f \leftrightarrow L \quad \text{where} \quad
L = \{w : f(w) = \text{accept}\}.
\]

\subsection*{Observation}
Languages may be finite or infinite, but a ``program'' is always a finite description.

\section*{Finite Automata}
A \textbf{deterministic finite automaton (DFA)} consists of:
\begin{itemize}
    \item States (nodes).
    \item Transitions labeled by alphabet symbols.
    \item Unique start state $q_0$.
    \item Accept states (double circles).
\end{itemize}

\begin{definition}
A DFA is a 5-tuple $M = (Q, \Sigma, \delta, q_0, F)$ where:
\begin{itemize}
    \item $Q$ = finite set of states
    \item $\Sigma$ = alphabet
    \item $\delta : Q \times \Sigma \to Q$ = transition function
    \item $q_0 \in Q$ = start state
    \item $F \subseteq Q$ = accepting states
\end{itemize}
\end{definition}

\begin{definition}
The extended transition function $\delta^* : Q \times \Sigma^* \to Q$ is defined by:
\[
\delta^*(q,\epsilon) = q, \quad
\delta^*(q, w a) = \delta(\delta^*(q, w), a).
\]
\end{definition}

\newpage
\section{1/8 Functional Programming}
What is functional programming?
\\
When we write software and it gets complicated, it helps to break it down into smaller pieces. With functional programming, we get 'glue' where you are composing functions with each other.
\\
Goal: get us to write a fizzbuzz function that takes in an int and gives a fizzbuzz string output
\\
\begin{lstlisting}
  function fizzbuzz:
    input n number
    output string

    if (n % 3) = (n % 5) = 0 return "fizzbuzz"
    else if n % 3 = 0 return "fizz"
    else if n % 5 = 0 return "buzz"
    else return n as a string
\end{lstlisting}
\subsection{OCaml notes}
\begin{itemize}
  \item in the REPL (utop) terminate expressions with ;;
  \item standard arithmetic, carrot is string concat
  \item must use +. to add floats
  \item use float of int to cast float to int
  \item double quotes for string
  \item not binds tighter than other
  \item if false then 2 else 3 (ternerary must have same type)
  \item comparison operators, single =
  \item let x = 4 in x / 2
  \item overwriting variables instead of mutating them
  \item \begin{lstlisting}
    let <var> = <expr> in <body> ;;
    < store, let <var> = <expr>> -> <store[<var> ;;
  \end{lstlisting}
\end{itemize}





\newpage
\section{1/12}
\begin{lstlisting}
  let x = 19 in let x = x < 10 in x ;;
  (* goes to *)
  
  
  let x = 3 in x + x ;; (* -> 6 *)
  let x = 4 in let y = 5 in x * y ;; (* 20 *)
  let x = 19 in (let x = x < 10 in x) ;;
\end{lstlisting}

\subsection{Functions}
Functions are boring and everywhere.
\begin{lstlisting}
  (* anonymous function *)
  let f = fun (x : int) -> x + 1 ;;
  f : int -> int -> int
\end{lstlisting}
OCaml is right-associative so 
\begin{lstlisting}
  let f = fun (x : int) -> fun (y : int) -> x * y ;;
  let f (x : int) (y : int) = x * y ;;
\end{lstlisting}
These functions are isomorphic and will result in about the same thing.
\begin{lstlisting}
  let f (g : int -> int) (y : int) = g y;;
  let adder = f (fun (y : int) )
\end{lstlisting}
Lexical Scope
\begin{lstlisting}
  let y   = 1 in
  let f x = x + y in
  let y   = 2 in 
  f 3 ;;
\end{lstlisting}
There is closure used where the function substitutes y into its body and just becomes f(x) = x + 1.

\subsection{Specification}
We have high-order functions that can take in other functions.
\begin{lstlisting}
  let max3 (i : int) (j : int) (k : int) = 
    if i >= j
    then if i >= k
      then i
      else k
    else if l >= k
      then j
      else k
\end{lstlisting}
We want to specify the function in formal logic.
\begin{multline*}
  \forall i, j, k \in \mathbb{Z} \\
    max3(i, j, k) \geq i \land \\
     max3(i,j,k) \geq j \land \\
     max3(i,j,k) \geq k \land \\
     max3(i,j,k) \in \{i, j, k\}
\end{multline*}

\newpage
\section{January 20, Divide and Conquer cont.}
Quiz 1 1/21 - Basic iterative and recursive algorithms (study first 2 lectures). Practice quiz on canvas.
\\
Homework 1 due on friday.
\subsection{Asymptotic Analysis cont.}
\begin{question}
  Rank the following functions in order of growth:
  \begin{enumerate}
    \item $n \log_2 n$
    \item $n^2$
    \item $100n$
    \item $3^{\log_2 n}$
  \end{enumerate}
  Starting with the first 2. 
  $\lim_{n \to \infty}\frac{\log_2n}{n} = \lim_{n \to \infty}\frac{\ln n}{\ln 2 \cdot n}$.
  \\
  Apply L'Hopital
  \\
  $\lim_{n \to \infty}\frac{1/n}{\ln 2} = \lim_{n \to \infty}\frac{1}{n \cdot \ln 2} = 0. \\
  \therefore \log_2 n = O(n)$
  \\
  $3^{\log_2 n} = (2^{\log_2 3})^{\log_2 n} = (2^{\log_2 n})^{\log_2 3} = n ^{\log_2 3} = n^{1.9}$
  \\
  \textbf{Answer:}
  \begin{enumerate}
    \item $100n$
    \item $n \log_2 n$
    \item $3^{\log_2 n}$
    \item $n^2$
  \end{enumerate}
\end{question}
h
\\
Big Oh Rules:
\begin{itemize}
  \item Constant factors can be ignored. $\forall C > 0, Cn = \mathcal{O}(n)$
  \item Lower order terms can be dropped. $n^2 + n^{3/2} + n = \mathcal{O}(n^2)$
  \item Smaller exponents are Big-Oh of larger exponents. $\forall a < b, n^a = \mathcal{O}(n^b)$
  \item Any logarithm is Big-Oh of any polynomial. $\forall a, b > 0, (log_2 n)^a = \mathcal{O}(n^b)$
  \item Any polynomial is Big-Oh of any exponential. $\forall a > 0, b > 1, n^a = \mathcal{O}(b^n)$
  \item Bases of logarithms can be ignored. $\forall a, b > 1, \log_a (n) = \mathcal{O}(\log_b (n))$
\end{itemize}
\begin{itemize}
  \item Big-Omega
\end{itemize}
\newpage
\section{1/15 Recap and Structural Recursion}
\begin{lstlisting}
type int_result = 
  | IntOk of int
  | IntError of string ;;
\end{lstlisting}
\begin{lstlisting}
type person = { name : string ; age : int; registered : bool} ;;
let alice = { name = "alice" ; age = 24 ; registered = true } ;;

let string_of_person (p : person) : string =
  p.name ^ " is " string_of_int p.age ;;
  
let other_string_of_person { name ; age = their_age ; _ } : string =
  name ^ " is " string_of_int their_age ;;
\end{lstlisting}
The million dollar mistake: nullity. Can be solved by options.
\begin{lstlisting}
type int_option =
  | IntOpt of int
  | IntOptEmpty
\end{lstlisting}

\subsection{Recursive types}
Lists in OCaml
\begin{lstlisting}
  [1 ; 2 ; 3]
\end{lstlisting}
Statically typed, can only have single type.

\begin{lstlisting}
  type int_list = 
    | Empty
    | Cons of int * int_list
  
  let intlist1 = Cons (2, (Cons (1, Empty)))

  let rec int_with_list_sum (xs : int_list) : int =
    match xs with
    | Empty -> 0
    | Cons (x, xs') -> x + int_list_sum xs'

  let rec int_list_length (xs : int_list) : int =
    | Empty -> 0
    | Cons (_x, xs) -> 1 + int_list_length xs
  
  let rec int_list_add1 (xs : int_list) : int_list =
    | Empty -> Empty
    | Cons (x, xs) -> Cons (x * 2, int_list_add1 xs)
  
  let rec int_list_mul2 (xs : int list) : int list =
    | [] -> []
    | x::xs -> (x * 2) :: (int_list_mul2 xs)
\end{lstlisting}
We can create a high-order function map for this
\begin{lstlisting}
  let rec map (f : int -> int) (xs : int list) : int list =
    match xs with 
    | [] -> []
    | x::xs -> (f x) :: (map f xs)
    
  let add1 = map (fun (x : int) -> x + 1)
  let mul2 = map (fun (x : int) -> x * 2)
\end{lstlisting}
We can reduce or fold to find the sum of a list. We are taking a data structure and an accumulator and apply a binary function to the acc and each element of the structure.
\begin{lstlisting}
  let rec fold ( f : int -> int -> int) (acc : int) (xs : int list) : int_list =
    match xs with
    | [] -> acc
    | (x::xs') ->
      let acc' = f acc x in
      fold f acc' xs'
  
  let sum = fold (fun (acc : int) (x : int) -> acc + x) 0
  let product = fold (fun (acc : int) (x : int) -> acc * x) 1

  let rec append (xs : int list) (ys : int list) : int list =
    match xs with
    | [] -> ys
    | (x :: xs') -> x :: append xs' ys

  let rec reverse (xs : int list): int list =
    match xs with
    | [] -> []
    | (x :: xs') -> append (reverse xs') [x]
\end{lstlisting}
Functions sum and reverse are commutative. $\forall x \in$ int list, sum(reverse(xs)) = sum(xs)
\\
How to test a list function.
\\
Generating a random list: start with deciding the length.
\begin{lstlisting}
  let rec mk_list_of_length (len : int) : int list =
    if len <= 0 then [] else
      let hd = Random_int 500 in
      let tl = mk_list_of_length (len - 1) in
      hd :: tl

  let list_gen () : int list =
    let len = Random.int 300 in
    mk_list_of_length len
\end{lstlisting}
We need to serialize a counterexample to create a tester
\begin{lstlisting}
  let prop_sum_rev (sum L int list -> int) (rev : int list -> int list) : int list option =
  let xs = list_gen () in
  if sum (reverse xs) = sum xs then None else 
    Some xs

  let rec do_tests (sum : int list -> int) (rev : int list -> int list) (n : int) =
    if  n <= 0 then
      print_endline "no issues"
    else
      match (prop_sum_rev sum rev) with
      | None -> do_tests sum rev (n - 1)
      | Some s -> print_endline("found a bug: " ^ stringify s)
\end{lstlisting}

\end{document}
