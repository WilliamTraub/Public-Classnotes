\section{Number Sets}
A few famous sets are:
\begin{itemize}
  \item $\varnothing$: the \textit{empty set} containing no elements;
  \item $\mathbb{N}$: the set of \textit{natural numbers} (that is, nonnegative integers);
  \item $\mathbb{Z}$: the set of \textit{integers};
  \item $\mathbb{Q}$: the set of \textit{rational numbers};
  \item $\mathbb{R}$: the set of \textit{real numbers};
  \item $\mathbb{C}$: the set of \textit{complex numbers}.
\end{itemize}
\section{Definitions}
\begin{itemize}
  \item Given universal set U and proposition schema P(x)
  \begin{itemize}
    \item $\neg(\forall x \in U, P(x)) = \exists x \in U, \neg P(x)$
    \item $\neg(\exists x \in U, P(x)) = \forall x \in U, \neg P(x)$
  \end{itemize}
  \item A \textbf{relation} from set X to set Y is a subset $R \subseteq X \times Y$.
  \item A \textbf{function} with domain X and codomain Y is a relation such that:
  \begin{itemize}
    \item $\forall x \in X, \exists y \in Y$
  \end{itemize}
  \item A \textbf{Model of the second-order formulation of peano arithmetic} is a 3-tuple (N, O, S) where N is a set, $O \in N$, and $S: N \to N$ which all satisfy the second-order peano axioms. All models like this are isomorphic.
\end{itemize}