\section{Functions}
\subsection{Definitions}
\begin{definition}
  A \textbf{Function} is defined by a subset of $A \times B$: 
  \[\Gamma_f := \{ (a, b) \in A \times B | b = f(a)\} \subseteq A \times B \]
  This set $\Gamma_f$ is the \textit{graph} of f; a function is fully represented by its graph.
  \\
  Functions are required to follow $(\forall a \in A)(\exists !b \in B) \: f(a) = b$
\end{definition}
Identity function: $id_A : A \to A$ or $(\forall a \in A) \: id_A(a) = a$ 


\subsection{Indexed Sets}
An indexed set $\{a_i \}_{i \in I}$ is informaly defined as $a_i$ for i ranging over some set of indicies I. The more formal definition is a function $I \to A$ where A is some set from which we draw the elements $a_i$.


\subsection{Composition of functions}
Functions may be composed if $f : A \to B$ and $g : B \to C$ are functions, then so is the operation $g \circ f$ defined by:
\[ (\forall a \in A) \: (g \circ f)(a) := g(f(a)) \]
Composition is commutative and associative.


\subsection{Injections, surjections, bijections}
\begin{itemize}
  \item A function $f: A \to B$ is \textit{injective} if $(\forall a' \in A)(\forall a'' \in A) \: a' \neq a'' \implies f(a') \neq f(a'')$. That is, if f sends different elements to different elements.
  \item A function $f: A \to B$ is \textit{surjective} if $(\forall b \in B)(\exists a \in A) \: b = f(a)$. That is, if f 'covers the whole of B' (im f = b)
\end{itemize}
Injections are often drawn $\hookrightarrow$; surjections are often drawn $\twoheadrightarrow$.
\\
If f is both injective and surjective, we say it is \textit{bijective} or and \textit{isomorphism of sets}. Where we write $\cong$


\subsection{Injections, surjections, bijections: Second viewpoint}
If $f : A \to B$ is a bijection, than we can 'flip its graph' to define a function $g : B \to A$.
\\
Assume $A \neq \emptyset$, and let $f: A \to B$ be a function:
\begin{enumerate}
  \item f has a left-inverse if and only if it is injective.
  \item f has a right-inverse if and only if it is surjective.
\end{enumerate}
This implies a function $f : A \to B$ if a bijection if and only if it has a two-sided inverse. ??.
\subsection{Monomorphisms and epimorphisms}
There is another way to express injectivity and surjectivity.
\\
A function $f: A \to B$ is a \textit{monomorphism} if the following holds:
\\
$\textit{for all sets Z and all functions } \alpha', \alpha'' : Z \to A$
\\
$f \circ \alpha' = f \circ \alpha'' \implies \alpha' = \alpha''$


\subsection{Excercises}
\begin{excercise}
  Prove that the inverse of a bijection is a bijection and that the composition of two bijections is a bijection.
  \\
  First to prove the inverse of a bijective function f is injective:
  \\
  By Proposition 2.1, b
\end{excercise}