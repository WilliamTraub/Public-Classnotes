\section{October 14 - Turing Machine Variants}
\subsection*{Multi-Tape TM}
A TM with one input tape and multiple work tapes. Transition function is defined by $\delta = Q' \times T \to Q \times T \times \{L, R\}$.   $\delta(g, w) = (g', w, L/R)$.
\begin{itemize}
  \item You can concatenate multiple tapes to one tape and separate their contents by \#. 
  \item Remember tape-head positions by storing an \underline{underlined} version of tape symbols. 
  \item Each step of multi-tape TM is simulated by scanning entire tape of single-tape TM.
  \item If you run out of space on the tapes, shift all elements to the right. (halting problem)
\end{itemize}
Tape-Head Level Description:
\\
f(a, b) = a + b. Input (binary interpretations of two integers separated by a \#)
\begin{itemize}
  \item Reverse each input and copy each one to a different tape and clear main tape. 
  \item Return all tape heads to the left. Store 1-bit carry as 0
  \item Add two bits under each tape head (using 0 if one head is empty) and carry bit:
    \begin{itemize}
      \item Write result mod 2 to the main tape.
      \item Move all tape heads 1 right.
      \item Repeat until both heads are empty.
    \end{itemize}
  \item Reverse main tape.
\end{itemize}

\subsection*{Random Access TM}
Can read/write to arbitrary locations in memory without scanning a tape. Memory modeled as infinite array R.
\begin{itemize}
  \item In addition to the standard tape that contains the input the TM has location and value tapes.
  \item There is a special write transition which sets R[location] = value using the content of the tapes.
  \item There is a read transition which sets the contents of the values tape to R[location]
\end{itemize}
Compliling to normal TM:
\\
We use a multi-tape machine (which can be converted to a single-tape)
\begin{itemize}
  \item Store contents of array R on a tape as tuples (location$_n$, value$_n$).
  \item To simulate a read, scan R until find a location that matches content of location tape. Write the value on the value tape. Put a blank if no such value is found.
  \item To simulate a write, scan R until you find location that matches content of location tape. Update value. If none found, append (location, value) to end of R.
\end{itemize}
\subsection*{Turing Completeness}
\begin{theorem}
  Church-Turing Thesis: Any algorithm (in an informal sense) can be computed by a TM.
\end{theorem}
Proof Outline:
\\
Design a compiler that converts Java program into a TM.
\begin{itemize}
  \item All programming languages are already compiled to "assembly code" for modern CPUs
  \item Assembly code instructions can be implemented on a Random-Access TM.
\end{itemize}

