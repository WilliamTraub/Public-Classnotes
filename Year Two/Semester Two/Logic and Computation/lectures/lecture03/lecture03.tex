\section{1/12}
\begin{lstlisting}
  let x = 19 in let x = x < 10 in x ;;
  (* goes to *)
  
  
  let x = 3 in x + x ;; (* -> 6 *)
  let x = 4 in let y = 5 in x * y ;; (* 20 *)
  let x = 19 in (let x = x < 10 in x) ;;
\end{lstlisting}

\subsection{Functions}
Functions are boring and everywhere.
\begin{lstlisting}
  (* anonymous function *)
  let f = fun (x : int) -> x + 1 ;;
  f : int -> int -> int
\end{lstlisting}
OCaml is right-associative so 
\begin{lstlisting}
  let f = fun (x : int) -> fun (y : int) -> x * y ;;
  let f (x : int) (y : int) = x * y ;;
\end{lstlisting}
These functions are isomorphic and will result in about the same thing.
\begin{lstlisting}
  let f (g : int -> int) (y : int) = g y;;
  let adder = f (fun (y : int) )
\end{lstlisting}
Lexical Scope
\begin{lstlisting}
  let y   = 1 in
  let f x = x + y in
  let y   = 2 in 
  f 3 ;;
\end{lstlisting}
There is closure used where the function substitutes y into its body and just becomes f(x) = x + 1.

\subsection{Specification}
We have high-order functions that can take in other functions.
\begin{lstlisting}
  let max3 (i : int) (j : int) (k : int) = 
    if i >= j
    then if i >= k
      then i
      else k
    else if l >= k
      then j
      else k
\end{lstlisting}
We want to specify the function in formal logic.
\begin{multline*}
  \forall i, j, k \in \mathbb{Z} \\
    max3(i, j, k) \geq i \land \\
     max3(i,j,k) \geq j \land \\
     max3(i,j,k) \geq k \land \\
     max3(i,j,k) \in \{i, j, k\}
\end{multline*}
