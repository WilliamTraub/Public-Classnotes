\documentclass[12pt]{article}

\usepackage[normalem]{ulem}
\newcommand{\mydash}{\hbox{\sout{ }}} 
\renewcommand{\arraystretch}{1.25}

\usepackage{amsmath, amssymb, amsthm}
\usepackage{tikz}
\usepackage{hyperref}
\usepackage[a4paper, margin=1in]{geometry}
\usepackage{tikz}
\usepackage{amsfonts}
\usepackage{latexsym}
\usepackage{array}
\usepackage{listings}
\usepackage{color}
\usepackage{marvosym}

\definecolor{dkgreen}{rgb}{0,0.6,0}
\definecolor{gray}{rgb}{0.5,0.5,0.5}
\definecolor{mauve}{rgb}{0.58,0,0.82}

\lstset{frame=tb,
  language=Java,
  aboveskip=3mm,
  belowskip=3mm,
  showstringspaces=false,
  columns=flexible,
  basicstyle={\small\ttfamily},
  numbers=none,
  numberstyle=\tiny\color{gray},
  keywordstyle=\color{blue},
  commentstyle=\color{dkgreen},
  stringstyle=\color{mauve},
  breaklines=true,
  breakatwhitespace=true,
  tabsize=3
}


\title{Algorithms and Data Notes}
\author{William Traub}
\date{}

\newtheorem{corollary}{Corollary}
\newtheorem{definition}{Definition}
\newtheorem{theorem}{Theorem}
\newtheorem{lemma}[theorem]{Lemma}
\newtheorem{example}{Example}
\newtheorem{observation}{Observation}
\newtheorem{question}{Question}
\newtheorem{exercise}{Exercise}


\begin{document}
\begin{exercise}
Mystery Function
\begin{lstlisting}
  Function F(a, n):
    If n = 0 : Return (1, 0)
    Else:
      b <- 0
      For i from 1 to n:
        b <- b + a
      (u, v) <- F(a, n - 1)
      Return (u * b, v + n * b)
\end{lstlisting}
\begin{itemize}
  \item What are the results of
  \begin{itemize}
    \item F(a, 2) : $\langle 2a^2, 5a \rangle$
    \item F(a, 3)
    \item F(a, 4)
  \end{itemize}
  \item What does the code do?
  \begin{proof}
    Hypothesis: F(a, n) = $(n!n^n, \frac{n(n+1)(2n+1)a}{6})$
    Base Case: Substituting n=1, F(a, 1) = $(1!1^1, \frac{1(1+1)(2(1)+1)a}{6}) \to (1, 0)$
    Induction Step:
    \\
    If statement proven by base case.
    \\
    $b = n \cdot a$
    \\
    (u,v) = F(a, n - 1)
    \\
    $u = (n-1)!a^{n-1}$
    \\
    $v = \frac{(n-1)(n)(2(n-1) + 1)a}{6}$
    \\
    Return statement (x, y):
    \begin{itemize}
      \item 
    \end{itemize}
  \end{proof}
  
\end{itemize}
\end{exercise}
\begin{exercise}
  Tiling with tetrominoes.
  \\
  \begin{proof}
    \begin{coroll