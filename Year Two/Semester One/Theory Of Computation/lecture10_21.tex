\section{October 21}
For each object O, let $<O>$ be a string that encodes O. 
\begin{itemize}
    \item If i is an integer $<i>$ can be its representation
    \item If O is a string, then $<O>$ is just O itself.
    \item If G is a graph $<G>$ can be defined in many ways; such as vertex lists, etc
    \item If M is a TM, we can define $<M>$ by writing down the formal definition. $M = (Q, \Sigma, \Gamma, \delta, q_{start}, q_{accept}, q_{reject})$
\end{itemize}
Now we will consider more complex languages, i.e. primes, graphs, etc.

\subsection{Universal Turing Machine}
\begin{itemize}
  \item There is a Turing Machine $M_{\text{UNIV}}$ that can run any other TM.
  \item $M_{\text{UNIV}}(<M>, w)$. Takes as input a description of any TM M and any string W. Runs M on w.
  \begin{itemize}
    \item If M accepts w, then $M_{\text{UNIV}}(<M>, w)$ accepts.
    \item If M rejects w, then $M_{\text{UNIV}}(<M>, w)$ rejects.
    \item If M loops on w, then $M_{\text{UNIV}}(<M>, w)$ also loops.
  \end{itemize}
  \item TMs are algorithms and a universal TM is a general purpose computer.
\end{itemize}

